\label{ch:bijlagen}

\newpage
\begin{lstlisting}[
frame=single,
breaklines=true,
caption=VagrantFile,
language=Ruby,
basicstyle=\ttfamily,
keywordstyle=\color{red},
stringstyle=\color{blue},
]
# -*- mode: ruby -*-
# vi: ft=ruby :

require 'rbconfig'
require 'yaml'

# Set your default base box here
DEFAULT_BASE_BOX = 'bento/centos-7.4'

VAGRANTFILE_API_VERSION = '2'
PROJECT_NAME = '/' + File.basename(Dir.getwd)

# When set to `true`, Ansible will be forced to be run locally on the VM
# instead of from the host machine (provided Ansible is installed).
FORCE_LOCAL_RUN = false

$ITERATIONS = 0
hosts = YAML.load_file('vagrant-hosts.yml')

# {{{ Helper functions
def provision_ansible(config)

  if run_locally?
    # Provisioning configuration for shell script.
    #The Iteration is here to prevent the provisioning from looping, as the command is being called inside a loop this stops it from being executed every call.
    if $ITERATIONS == 0
    config.vm.provision 'shell' do |sh|
      sh.path = 'scripts/playbook-win.sh'
    $ITERATIONS = 1
    end
  end
  else
    # Provisioning configuration for Ansible (for Mac/Linux hosts).
    config.vm.provision 'ansible' do |ansible|
      ansible.playbook = 'ansible/site.yml'
      ansible.sudo = true
    end
  end
end

def run_locally?
  windows_host? || FORCE_LOCAL_RUN
end

def windows_host?
  RbConfig::CONFIG['host_os'] =~ /mswin|mingw|cygwin/
end

# Set options for the network interface configuration. All values are
# optional, and can include:
# - ip (default = DHCP)
# - netmask (default value = 255.255.255.0
# - mac
# - auto_config (if false, Vagrant will not configure this network interface
# - intnet (if true, an internal network adapter will be created instead of a
#   host-only adapter)
def network_options(host)
  options = {}

  if host.key?('ip')
    options[:ip] = host['ip']
    options[:netmask] = host['netmask'] ||= '255.255.255.0'
  else
    options[:type] = 'dhcp'
  end

  options[:mac] = host['mac'].gsub(/[-:]/, '') if host.key?('mac')
  options[:auto_config] = host['auto_config'] if host.key?('auto_config')
  options[:virtualbox__intnet] = true if host.key?('intnet') && host['intnet']
  options
end

def custom_synced_folders(vm, host)
  return unless host.key?('synced_folders')
  folders = host['synced_folders']

  folders.each do |folder|
    vm.synced_folder folder['src'], folder['dest'], folder['options']
  end
end

# }}}

Vagrant.configure(VAGRANTFILE_API_VERSION) do |config|
  config.ssh.insert_key = false
#   VyOS Router, comment line 89-104 if you want the PFSense router/Firewall to be build.
#  config.vm.define 'router' do |router|
#    router.vm.box = 'bertvv/vyos116'
#    router.vm.network :private_network,
#      ip: '192.0.2.254',
#      netmask: '255.255.255.0',
#      auto_config: false
#    router.vm.network :private_network,
#      ip: '172.16.255.254',
#      netmask: '255.255.0.0',
#      auto_config: false
#    router.ssh.insert_key = false

#    router.vm.provision "shell" do |sh|
#      sh.path = "scripts/router-config.sh"
#    end
#  end
  #PFSense firewall, comment line 106 -  127 If you want the VYoS router to be build.
  config.vm.define 'firewall' do |firewall|
    firewall.vm.box = 'kennyl/pfsense'
    firewall.ssh.username = 'vagrant'
    firewall.ssh.password = 'vagrant'
    firewall.ssh.shell = '/bin/sh'
    firewall.vm.guest = 'freebsd'
    firewall.vm.synced_folder ".", "/vagrant", disabled: true
    firewall.vm.network :private_network,
      ip: '172.16.255.254',
      netmask: '255.255.0.0',
      auto_config: false
    firewall.ssh.insert_key = false

    firewall.vm.provision "file" do |fil|
      fil.source = "scripts/FirewallScript"
      fil.destination = "/home/vagrant/"
    end
    
    firewall.vm.provision "shell" do |sh|
      sh.path = "scripts/FirewallScript/firewall-config.sh"
    end
  end
  hosts.each do |host|
    config.vm.define host['name'] do |node|
      node.vm.box = host['box'] ||= DEFAULT_BASE_BOX
      node.vm.box_url = host['box_url'] if host.key? 'box_url'

      node.vm.hostname = host['name']
      node.vm.network :private_network, network_options(host)
      custom_synced_folders(node.vm, host)

      node.vm.provider :virtualbox do |vb|
        # WARNING: if the name of the current directory is the same as the
        # host name, this will fail.
        vb.customize ['modifyvm', :id, '--groups', PROJECT_NAME]
        vb.memory = 512
      end
      #This function is placed inside this loop to prevent it from being called on the router and firewall
      provision_ansible(config)
    end
  end
end
\end{lstlisting}

\newpage


\begin{lstlisting}[
frame=single,
breaklines=true,
caption=Vagrant-hosts.yml,
language=Ruby,
basicstyle=\ttfamily,
keywordstyle=\color{red},
stringstyle=\color{blue},
]
# vagrant_hosts.yml
#
# List of hosts to be created by Vagrant. This file controls the Vagrant
# settings, specifically host name and network settings. You should at least
# have a `name:`.

---
- name: samba
ip: 172.16.0.11
netmask: 255.255.0.0

- name: dns
ip: 172.16.0.10
netmask: 255.255.0.0

- name: dhcp
ip: 172.16.0.12
netmask: 255.255.0.0


\end{lstlisting}
\begin{lstlisting}[
frame=single,
breaklines=true,
caption=site.yml,
language=Ruby,
basicstyle=\ttfamily,
keywordstyle=\color{red},
stringstyle=\color{blue},
]
# site.yml
---
- hosts: all
become: true
roles:
- bertvv.rh-base

- hosts: dns
become: true
roles:
- bertvv.bind

- hosts: dhcp
become: true
roles:
- bertvv.dhcp

- hosts: samba
become: true
roles:
- bertvv.samba
- bertvv.vsftpd
post_tasks:
- name: teachers all access in ExaminationFiles
acl:
path: /srv/shares/ExaminationFiles
entity: teachers
etype: group
permissions: rwx
state: present
default: yes
- name: students read access in ExaminationFiles
acl:
path: /srv/shares/ExaminationFiles
entity: students
etype: group
permissions: rx
state: present
default: yes
- name: students prohibit access in Teachers folder
acl:
path: /srv/shares/Teachers
entity: students
etype: group
permissions: ---
state: present
default: yes

\end{lstlisting}
\begin{lstlisting}[
frame=single,
breaklines=true,
caption='group\textunderscore vars/all.yml',
language=Ruby,
basicstyle=\ttfamily,
keywordstyle=\color{red},
stringstyle=\color{blue},
]
# group_vars/all.yml
# Variables visible to all nodes
---

rhbase_install_packages:
- bash-completion
- bind-utils
- git
- nano
- tree
- vim-enhanced
- wget
rhbase_users:
- name: kjell
comment: 'kjell admin'
groups:
- users
- wheel
password: 'KEY'

rhbase_motd: true
rhbase_ssh_user: kjell
rhbase_ssh_key: ssh-rsa SSH

\end{lstlisting}
\begin{lstlisting}[
frame=single,
breaklines=true,
caption=dhcp.yml,
language=Ruby,
basicstyle=\ttfamily,
keywordstyle=\color{red},
stringstyle=\color{blue},
]
# host_vars/pu001.yml
# Variables visible to pr001
---
rhbase_firewall_allow_services:
- dhcp

dhcp_global_authoritative: authoritative
dhcp_global_default_lease_time: 43200
dhcp_global_domain_name: example.lan
dhcp_global_routers: 172.16.255.254
dhcp_global_domain_name_servers:
- 172.16.255.254
dhcp_global_classes:
- name: vbox
match: 'match if binary-to-ascii(16,8,":",substring(hardware, 1, 3)) = "8:0:27"'
- name: reserved
match: 'match if hardware = 1:00:11:22:33:44:55'

dhcp_subnets:
- ip: 172.16.0.0
netmask: 255.255.0.0
pools:
- range_begin: 172.16.0.2
range_end: 172.16.127.254
- range_begin: 172.16.128.1
range_end: 172.16.191.254
allow: 'members of "reserved"'
- range_begin: 172.16.192.1
range_end: 172.16.255.253
allow: 'members of "vbox"'

\end{lstlisting}
\begin{lstlisting}[
frame=single,
breaklines=true,
caption=dns.yml,
language=Ruby,
basicstyle=\ttfamily,
keywordstyle=\color{red},
stringstyle=\color{blue},
]
rhbase_firewall_allow_services:
- dns
bind_zone_master_server_ip: 172.16.0.10

bind_zone_name: example.lan
bind_zone_networks:
- '172.16.0'
bind_zone_name_servers:
- dns
bind_zone_hosts	:
- name: dns
ip: 172.16.0.10
aliases:
- nameserver
- name: samba
ip: 172.16.0.11
aliases:
- files
- name: dhcp
ip: 172.16.0.12
aliases:
- internal

bind_listen_ipv4:
- any
bind_allow_query:
- any
bind_forwarders:
- '172.16.0.10'

\end{lstlisting}
\begin{lstlisting}[
frame=single,
breaklines=true,
caption=samba.yml,
language=Ruby,
basicstyle=\ttfamily,
keywordstyle=\color{red},
stringstyle=\color{blue},
]
rhbase_firewall_allow_services:
- samba
- ftp
rhbase_user_groups:
- students
- teachers
- it
- public

rhbase_users:
- name: kjell
comment: 'kjell admin'
groups:
- wheel
- it
- public
password: 'KEY'
- name: docent
comment: 'docent'
groups:
- public
- teachers
password: 'KEY'
- name: student
comment: 'student'
groups:
- public
- students
password: 'KEY'
shell: /sbin/nologin
- name: student2
comment: 'student2'
groups:
- public
- students
password: 'KEY'
shell: /sbin/nologin
samba_users:
- name: kjell
password: test
- name: docent
password: teacher
- name: student
password: student
samba_shares:
- name: HandIn
comment: 'Share folder to hand in examinations'
group: public
valid_users: +students,+teachers,+it
write_list: +students,+teachers,+it
create_mode: '1775'
force_create_mode: '1775'
directory_mode: '1775'
force_directory_mode : '1775'
- name: ExaminationFiles
comment: 'Share folder where students can find the exams'
group: public
valid_users: +students,+teachers,+it,+public
write_list: +teachers
create_mode: '0775'
force_create_mode: '0775'
directory_mode: '0775'
force_directory_mode : '0775'

- name: Teachers
comment: 'Share folder for teachers, files with account info etc.'
group: teachers
valid_users: +teachers,+it
write_list: +teachers,+it
#directory_mode: '770'
samba_workgroup: example
samba_netbios_name: EXAMINATION
samba_load_homes: true
sambe_load_printers: false
vsftpd_anonymous_enable: false
vsftpd_local_enable: true
vsftpd_write_enable: true
vsftpd_local_umask: '007'
vsftpd_local_root: '/srv/shares/'
\end{lstlisting}
\begin{lstlisting}[
frame=single,
breaklines=true,
caption=inventory.py,
language=Ruby,
basicstyle=\ttfamily,
keywordstyle=\color{red},
stringstyle=\color{blue},
]
#! /usr/bin/python
# coding=utf8
#
# Inventory script for Ansible skeleton, to be used on Windows host systems

import socket
import sys
from getopt import getopt, GetoptError


#
# Helper functions
#


def usage():
print ("Usage: %s [OPTION]\n"
"  --list  list all hosts\n"
"  --host=HOST  gives extra info"
"about the specified host\n") % sys.argv[0]


def list_hosts():
host_name = socket.gethostname()
print "{ \"all\": { \"hosts\": [\"%s\"] } }" % host_name


def host_info(host):
print "{}"

#
# Parse command line
#

try:
opts, args = getopt(sys.argv[1:], "lh:", ['list', 'host='])
except GetoptError as err:
print str(err)
usage()
sys.exit(2)

for opt, opt_arg in opts:
if opt in ('-l', '--list'):
list_hosts()
sys.exit(0)
if opt in ('-h', '--host'):
host_info(opt_arg)
else:
assert False, "unhandled option: %s" % opt

\end{lstlisting}
\begin{lstlisting}[
frame=single,
breaklines=true,
caption=playbook-win.sh,
language=Ruby,
basicstyle=\ttfamily,
keywordstyle=\color{red},
stringstyle=\color{blue},
]
#!/bin/bash
# Source: https://github.com/geerlingguy/JJG-Ansible-Windows/blob/master/windows.sh

# Windows shell provisioner for Ansible playbooks, based on KSid's
# windows-vagrant-ansible: https://github.com/KSid/windows-vagrant-ansible
#
# @todo - Allow proxy configuration to be passed in via Vagrantfile config.
#
# @see README.md
# @author Jeff Geerling, 2014
# @version 1.0
#

#
# Bash shell settings: exit on failing commands, unbound variables
#
set -o errexit
set -o nounset
set -o pipefail

#
# Variables
#
ITERATIONS=0
# Color definitions
readonly reset='\e[0m'
readonly red='\e[0;31m'
readonly yellow='\e[0;33m'
readonly cyan='\e[0;36m'

readonly playbook=/vagrant/ansible/site.yml
readonly inventory=/vagrant/scripts/inventory.py

#
# Functions
#

main() {
info "Running Playbook"
if [ "$ITERATIONS" == 0 ]; then
info "Running Ansible playbook ${playbook} locally on host ${HOSTNAME}."

exit_on_vyos
check_if_playbook_exists
check_if_inventory_exists
ensure_ansible_installed
run_playbook "${@}"
fi
ITERATIONS=1
}

exit_on_vyos() {
# If we're on a VyOS box, this script shouldn't be executed
if user_exists vyos; then
debug "On VyOS, not running Ansible here"
exit 0
fi
}

check_if_playbook_exists() {
if [ ! -f ${playbook} ]; then
die "Cannot find Ansible playbook ${playbook}."
fi
}

check_if_inventory_exists() {
if [ ! -f ${inventory} ]; then
die "Cannot find inventory file ${inventory}."
fi
}

ensure_ansible_installed() {
if ! is_ansible_installed; then
distro=$(get_linux_distribution)
"install_ansible_${distro}" || die "Distribution ${distro} is not supported"
fi

info "Ansible version"
ansible --version
}

is_ansible_installed() {
which ansible-playbook > /dev/null 2>&1
}

run_playbook() {
info "Running the playbook"

# Get absolute path to playbook command
playbook_cmd=$(which ansible-playbook)

set -x
${playbook_cmd} "${playbook}" \
--inventory-file="${inventory}" \
--limit="${HOSTNAME}" \
--extra-vars "is_windows=true" \
--connection=local \
"$@"
set +x
}



# Print the Linux distribution
get_linux_distribution() {

if user_exists vyos; then

echo "vyos"

elif [ -f '/etc/redhat-release' ]; then

# RedHat-based distributions
cut --fields=1 --delimiter=' ' '/etc/redhat-release' \
| tr "[:upper:]" "[:lower:]"

elif [ -f '/etc/lsb-release' ]; then

# Debian-based distributions
grep DISTRIB_ID '/etc/lsb-release' \
| cut --delimiter='=' --fields=2 \
| tr "[:upper:]" "[:lower:]"

fi
}

# Install Ansible on a Fedora system.
# The version in the repos is fairly current, so we'll install that
install_ansible_fedora() {
info "Fedora: installing Ansible from distribution repositories"
dnf -y install ansible
}

# Install Ansible on a CentOS system from EPEL
install_ansible_centos() {
info "CentOS: installing Ansible from the EPEL repository"
yum -y install epel-release
yum -y install ansible
}

# Install Ansible on a recent Ubuntu distribution, from the PPA
install_ansible_ubuntu() {
info "Ubuntu: installing Ansible from PPA"
# Remark: on older Ubuntu versions, it's python-software-properties
apt-get -y install software-properties-common
apt-add-repository -y ppa:ansible/ansible
apt-get -y update
apt-get -y install ansible
}

# Checks if the specified user exists
user_exists() {
user_name="${1}"
getent passwd "${user_name}" > /dev/null
}

# Usage: info [ARG]...
#
# Prints all arguments on the standard output stream
info() {
printf "${yellow}>>> %s${reset}\n" "${*}"
}

# Usage: debug [ARG]...
#
# Prints all arguments on the standard output stream
debug() {
printf "${cyan}### %s${reset}\n" "${*}"
}

# Usage: error [ARG]...
#
# Prints all arguments on the standard error stream
error() {
printf "${red}!!! %s${reset}\n" "${*}" 1>&2
}

# Usage: die MESSAGE
# Prints the specified error message and exits with an error status
die() {
error "${*}"
exit 1
}

main "${@}"

\end{lstlisting}
\begin{lstlisting}[
frame=single,
breaklines=true,
caption=role-deps.sh,
]
#! /usr/bin/env bash
#
# Author: Bert Van Vreckem <bert.vanvreckem@gmail.com>
#
# Install role dependencies. This script is meant to be used in the context of
# ansible-skeleton[1] and ansible-role-skeleton[2].
#
# This script will search ansible/site.yml (or the specified playbook) for
# roles assigned to hosts in the form "user.role". It will then try to install
# them all. If possible, it will use Ansible Galaxy (on Linux, MacOS), but if
# this is not available (e.g. on Windows), it will use Git to clone the latest
# revision.
#
# Remark that this is a very crude technique and especially the Git fallback
# is very brittle. It will download HEAD, and not necessarily the latest
# release of the role. Additionally, the name of the repository is guessed,
# but if it does not exist, the script will fail.
#
# Using ansible-galaxy and a dependencies.yml file is the best method, but
# unavailable on Windows. This script is an effort to have a working
# alternative.
#
# [1] https://github.com/bertvv/ansible-skeleton
# [2] https://github.com/bertvv/ansible-role-skeleton

set -o errexit  # abort on nonzero exitstatus
set -o nounset  # abort on unbound variable

#{{{ Variables
readonly SCRIPT_NAME=$(basename "${0}")

readonly ansible_root=ansible
readonly playbook=${ansible_root}/site.yml
readonly roles_path=${ansible_root}/roles
#}}}

main() {
local dependencies

process_args "${@}"

set_roles_path
select_installer

dependencies="$(find_dependencies)"

for dep in ${dependencies}; do
owner=${dep%%.*}
role=${dep##*.}

if [[ ! -d "${roles_path}/${dep}" ]]; then
${installer} "${owner}" "${role}"
else
echo "+ Skipping ${dep}, seems to be installed already"
fi
done
}

#{{{ Helper functions

# If the default roles path does not exist, try "roles/"
set_roles_path() {
if [ ! -d "${roles_path}" ]; then
if [ -d "${ansible_root}" ]; then
mkdir "${roles_path}"
else
roles_path="roles"
fi
fi
}

# Find dependencies in the specified playbook
find_dependencies() {
grep '    - .*\..*' "${playbook}" \
| cut -c7- \
| grep --invert-match " " \
| sort --unique
}

# Check if command line arguments are valid
process_args() {
if [ "${#}" -gt "1" ]; then
echo "Expected at most 1 argument, got ${#}" >&2
usage
exit 2
elif [ "${#}" -eq "1" ]; then
if [ "${1}" = '-h' ] || [ "${1}" = '--help' ]; then
usage
exit 0
elif [ ! -f "${1}" ]; then
echo "Playbook `${1}` not found." >&2
usage
exit 1
else
playbook="${1}"
fi
elif [ "${#}" -eq "0" ] && [ ! -f "${playbook}" ]; then
cat << _EOF_
Default playbook ${playbook} not found. Maybe you should cd to the
directory above ${playbook%%/*}/, or specify the playbook.
_EOF_
usage
exit 1
fi
}

# Print usage message on stdout
usage() {
cat << _EOF_
Usage: ${SCRIPT_NAME} [PLAYBOOK]

Installs role dependencies found in the specified playbook (or ${playbook}
if none was given).

OPTIONS:

-h, --help  Prints this help message and exits

EXAMPLES:

$ ${SCRIPT_NAME}
$ ${SCRIPT_NAME} test.yml
_EOF_
}


# Usage: select_installer
# Sets the variable `installer`, the function to use when installing roles
# Try to use ansible-galaxy when it is available, and fall back to `git clone`
# when it is not.
select_installer() {
if which ansible-galaxy > /dev/null 2>&1 ; then
installer=install_role_galaxy
else
installer=install_role_git
fi
}

# Usage: is_valid_url URL
# returns 0 if the URL is valid, 22 otherwise
is_valid_url() {
local url=$1

curl --silent --fail "${url}" > /dev/null
}

# Usage: install_role_galaxy OWNER ROLE
install_role_galaxy() {
local owner=$1
local role=$2
ansible-galaxy install --roles-path="${roles_path}" \
"${owner}.${role}"
}

# Usage: install_role_git OWNER ROLE
install_role_git() {
local owner=$1
local role=$2

# First try https://github.com/OWNER/ansible-role-ROLE
local repo="https://github.com/${owner}/ansible-role-${role}"

if is_valid_url "${repo}"; then
git clone "${repo}" "${roles_path}/${owner}.${role}"
else
# If that fails, try https://github.com/OWNER/ansible-ROLE
git clone "https://github.com/${owner}/ansible-${role}" \
"${roles_path}/${owner}.${role}"
fi
}
#}}}

main "${@}"

\end{lstlisting}
\begin{lstlisting}[
frame=single,
breaklines=true,
caption=router-config.sh,
basicstyle=\ttfamily,
keywordstyle=\color{red},
stringstyle=\color{blue},
]
#! /usr/bin/env bash
#
# Author: Bert Van Vreckem <bert.vanvreckem@gmail.com>
#
# Install role dependencies. This script is meant to be used in the context of
# ansible-skeleton[1] and ansible-role-skeleton[2].
#
# This script will search ansible/site.yml (or the specified playbook) for
# roles assigned to hosts in the form "user.role". It will then try to install
# them all. If possible, it will use Ansible Galaxy (on Linux, MacOS), but if
# this is not available (e.g. on Windows), it will use Git to clone the latest
# revision.
#
# Remark that this is a very crude technique and especially the Git fallback
# is very brittle. It will download HEAD, and not necessarily the latest
# release of the role. Additionally, the name of the repository is guessed,
# but if it does not exist, the script will fail.
#
# Using ansible-galaxy and a dependencies.yml file is the best method, but
# unavailable on Windows. This script is an effort to have a working
# alternative.
#
# [1] https://github.com/bertvv/ansible-skeleton
# [2] https://github.com/bertvv/ansible-role-skeleton

set -o errexit  # abort on nonzero exitstatus
set -o nounset  # abort on unbound variable

#{{{ Variables
readonly SCRIPT_NAME=$(basename "${0}")

readonly ansible_root=ansible
readonly playbook=${ansible_root}/site.yml
readonly roles_path=${ansible_root}/roles
#}}}

main() {
local dependencies

process_args "${@}"

set_roles_path
select_installer

dependencies="$(find_dependencies)"

for dep in ${dependencies}; do
owner=${dep%%.*}
role=${dep##*.}

if [[ ! -d "${roles_path}/${dep}" ]]; then
${installer} "${owner}" "${role}"
else
echo "+ Skipping ${dep}, seems to be installed already"
fi
done
}

#{{{ Helper functions

# If the default roles path does not exist, try "roles/"
set_roles_path() {
if [ ! -d "${roles_path}" ]; then
if [ -d "${ansible_root}" ]; then
mkdir "${roles_path}"
else
roles_path="roles"
fi
fi
}

# Find dependencies in the specified playbook
find_dependencies() {
grep '    - .*\..*' "${playbook}" \
| cut -c7- \
| grep --invert-match " " \
| sort --unique
}

# Check if command line arguments are valid
process_args() {
if [ "${#}" -gt "1" ]; then
echo "Expected at most 1 argument, got ${#}" >&2
usage
exit 2
elif [ "${#}" -eq "1" ]; then
if [ "${1}" = '-h' ] || [ "${1}" = '--help' ]; then
usage
exit 0
elif [ ! -f "${1}" ]; then
echo "Playbook `${1}` not found." >&2
usage
exit 1
else
playbook="${1}"
fi
elif [ "${#}" -eq "0" ] && [ ! -f "${playbook}" ]; then
cat << _EOF_
Default playbook ${playbook} not found. Maybe you should cd to the
directory above ${playbook%%/*}/, or specify the playbook.
_EOF_
usage
exit 1
fi
}

# Print usage message on stdout
usage() {
cat << _EOF_
Usage: ${SCRIPT_NAME} [PLAYBOOK]

Installs role dependencies found in the specified playbook (or ${playbook}
if none was given).

OPTIONS:

-h, --help  Prints this help message and exits

EXAMPLES:

$ ${SCRIPT_NAME}
$ ${SCRIPT_NAME} test.yml
_EOF_
}


# Usage: select_installer
# Sets the variable `installer`, the function to use when installing roles
# Try to use ansible-galaxy when it is available, and fall back to `git clone`
# when it is not.
select_installer() {
if which ansible-galaxy > /dev/null 2>&1 ; then
installer=install_role_galaxy
else
installer=install_role_git
fi
}

# Usage: is_valid_url URL
# returns 0 if the URL is valid, 22 otherwise
is_valid_url() {
local url=$1

curl --silent --fail "${url}" > /dev/null
}

# Usage: install_role_galaxy OWNER ROLE
install_role_galaxy() {
local owner=$1
local role=$2
ansible-galaxy install --roles-path="${roles_path}" \
"${owner}.${role}"
}

# Usage: install_role_git OWNER ROLE
install_role_git() {
local owner=$1
local role=$2

# First try https://github.com/OWNER/ansible-role-ROLE
local repo="https://github.com/${owner}/ansible-role-${role}"

if is_valid_url "${repo}"; then
git clone "${repo}" "${roles_path}/${owner}.${role}"
else
# If that fails, try https://github.com/OWNER/ansible-ROLE
git clone "https://github.com/${owner}/ansible-${role}" \
"${roles_path}/${owner}.${role}"
fi
}
#}}}

main "${@}"

\end{lstlisting}
\begin{lstlisting}[
frame=single,
breaklines=true,
caption=firewall-config.sh,
language=Ruby,
basicstyle=\ttfamily,
keywordstyle=\color{red},
stringstyle=\color{blue},
]
#!/bin/sh
pkg install -y git
mv /home/vagrant/FirewallScript/config.xml /cf/conf/
echo "Rebooting now!"
reboot

\end{lstlisting}