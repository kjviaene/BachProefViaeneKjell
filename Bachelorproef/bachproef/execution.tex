%%=============================================================================
%% Methodologie
%%=============================================================================

\chapter{Execution}
\label{ch:execution}

\section{Vyos Router / ACL's}
\subsection{Introduction}
As mentioned in the State of the Arts chapter, Vyos does not implement Access control lists like cisco devices do. Instead there is a build in firewall application. We use this to mimic the ACL's that one would configure on a cisco device. Our goal is to block all external sites except the ones that may be accessed during the exam (in our examples we'll just be using the standard school website). Furthermore, the vyos router is implemented only as a router. No DHCP/firewall settings were configured outside of the scope of this thesis. But if this would be the case, these settings would not interfere with our configuration. As long as the correct rules are followed of course. 
\subsection{Installation of Vyos}
The installation of Vyos is nothing really special. To start off you'll need the image. You can pick one that meets your needs. In the case of this thesis the setup was automated however. Which was done by using vagrant and a Vyos box \textit{bertvv/vyos116} [SOURCE VYOS]. This however is pretty much the same as manually downloading the .iso file, making a bootable device with it, and installing it onto your device of choice. After the installation of Vyos it is advised  to reboot the machine. It should be quite obvious that the machine that you wish to use should have 2 working network cards installed. One as an inside interface and one as an outside interface card. More than two can of course be used (for a DMZ zone for instance) but it is not required and not touched upon here.
\subsection{Configuration of the router}
All of the configurations for the router are placed in a shell script. This script is ran by vagrant when the machine is provisioned. Remember that all these settings can be done manually just as easily, explanations here about commands are given from a general point of view. Nothing changes if one chooses to automate the process or do it manually. The first things that are configured are the interfaces and basic information about the router.\\

\begin{cisco}[title=Basic configuration]
configure
#
# Basic settings
#
set system host-name 'Router'
set system domain-name example.lan
#
# IP settings
#
set interfaces ethernet eth0 address dhcp
set interfaces ethernet eth0 description WAN
set interfaces ethernet eth1 address 172.16.255.254/16
set interfaces ethernet eth1 description inside
\end{cisco} \\
The first command puts us in the configuration mode, if this command is forgotten, all the following commands will fail and the router will be left unconfigured. Next up the host-name and the domain name are configured. Followed by the configuration of the interfaces. There are two interfaces configured, both Ethernet ports. \textit{Eth0} is the port which leads to the outside and \textit{eth1} is our inside port which is connected to our local network \textit{example.lan}. Our inside interface is configured as the default gateway for all the devices in our network and has been given the IP-Address \textit{172.16.255.254} as is common practice.The outside interface gets it IP from our ISP.\\
The following commands configure the Network Address Translation (NAT) for our interfaces. In most networks this will be a lot more complicated but for us the only thing that is happening is our inside network addresses that get masqueraded when exiting via the outside interface.
\begin{cisco}[title=NAT configuration]
set nat source rule 100 outbound-interface eth0
set nat source rule 100 source address '172.16.0.0/16'
set nat source rule 100 translation address masquerade
#
# Domain Name Service
#
set system name-server 10.0.2.3
set service dns forwarding system
set service dns forwarding domain example.lan server 172.16.0.10
set service dns forwarding listen-on 'eth1'
\end{cisco}\\
After the NAT configuration, the DNS settings are configured, which are quite important to us. The system name-server command sets the main server that is chosen when DNS queries are received. In this case \textit{10.0.2.3} is the address that has to be configured when using virtual box. In a physical network however, this can be set to any other DNS server that you know of, that is outside of your network. Following the system name-server we enable dns forwarding and all the queries which concern a device inside the `` example.lan`` network are forwarded to our local server. This server has been given the \textit{172.16.0.10} IP-Address.\\
When looking at this configuration you might think that there is an other configuration here that suits your situation better. You can for instance change the system name-server to your local DNS server, so that all your requests get send to your local DNS server. Then you could configure your local DNS to forward all requests to an outside DNS. But as it's very likely that the DNS server will just send those requests back to the router for him to route them to the outside, we skip a step by just immediately making the router doing it this way. There is however an opportunity here that you might have noticed. If we indeed set our sytem name-server as our local dns server we can choose which domains get forwarded to other servers. In this case we would not allow our local server to forward any requests and we would effectively filter our packets.
\begin{cisco}[title=Filtering using DNS forwarding]
#set system name-server 172.16.0.10
#set service dns forwarding system
#set service dns forwarding domain hogent.be server 10.0.2.3
#set service dns forwarding listen-on 'eth1'
\end{cisco}\\
When using this configuration we actually blocked all other websites except the ones of the `` hogent.be `` domain. This achieves the same as we will achieve with our ACL's and looks a lot simpler. But it is not as simple to transfer this to a Cisco device. And because we are not actually using Vyos in our real life situation, we will take a look at our firewall configuration next.\\
The firewall configuration is split up into two main parts. Configuring our groups and configuring our rules. The configuration of the groups is in itself divided into two parts as well. The fist part being the configuration of our network group:\\
\begin{cisco}[title=network group configuration]
set firewall group network-group INSIDE-NET network 172.16.0.0/16
\end{cisco}
Here the group ``INSIDE-NET`` is created. This group simply contains the inside network address range. This group holds the group of addresses of all the devices that you want to be targeted. So if there are 5 networks that need to follow certain rules, all five would be added to this group, each with its own IP-Range.\\
The next groups to be configured are the address groups and the port groups:\\
\begin{cisco}[title= address and port groups configuration]
set firewall group address-group SCHOOL-NET address 178.62.144.90
set firewall group address-group SCHOOL-NET address 193.190.173.131
# 
set firewall group port-group TCP-ACCESS port 80
set firewall group port-group TCP-ACCESS port 443
\end{cisco}
The address-group `` SCHOOL-NET`` contains the IP addresses of the websites that we want to allow access to. You can add as much addresses as you want to this group, as long as they are allowed. This group could be compared with a whitelist. One could also make a group in the same way but use it as a blacklist. This by adding all known addresses that need to be blocked and just add them to an other group which will be implemented in an other way.\\
The port-group `` TCP-ACCESS`` contains our TCP ports that we want to block all traffic to. Here ports 80 (HTTP) and 443 (HTTPS) were chosen to block all packets that wish to connect with a web server. Each group can be given a description as well, this by simply replacing the `` port ``, `` address `` and `` network `` keywords with the keyword `` description ``.  And following that with a description of choice (encapsulated by ` marks ). Once all the necessary groups are configured, the only thing left to do is implement these groups in the correct way. And defining the rules surrounding them.\\
\begin{cisco}[title= Configuring the firewall rules]
set firewall name ACCESS-CONTROL description 'Blocking unwanted sites'
set firewall name ACCESS-CONTROL default-action drop
set firewall name ACCESS-CONTROL rule 200 description `Deny any HTTP and HTTPS packets`
set firewall name ACCESS-CONTROL rule 200 action drop
set firewall name ACCESS-CONTROL rule 200 destination group port-group TCP-ACCESS
set firewall name ACCESS-CONTROL rule 200 source group network-group INSIDE-NET
set firewall name ACCESS-CONTROL rule 200 state established enable
set firewall name ACCESS-CONTROL rule 200 state related enable
set firewall name ACCESS-CONTROL rule 100 description `Allow access to school websites`
set firewall name ACCESS-CONTROL rule 100 action accept
set firewall name ACCESS-CONTROL rule 100 destination group address-group SCHOOL-NET
set firewall name ACCESS-CONTROL rule 100 state established enable
set firewall name ACCESS-CONTROL rule 100 state related enable
\end{cisco}
The following things are happening here:
\begin{itemize}
\item A description is given to our rule-set. When working with bigger networks this is a must. Otherwise one might drown in all the different configured rule-sets on each interface.
\item The default action for the rule-set is set to drop. This means that all packages that do not match a single statement will be dropped. Other choices here are:
\begin{itemize}
\item Reject: Drops and notifies the source if no rules were matched.
\item Accept: Allows the packages that did not match a single rule to pass.
\end{itemize}
\item The declaration of a rule. Which consists out of multiple lines:
\begin{itemize}
\item The number of the rule. The lowest number will be checked first. So if an ``allow all`` rule would be inserted at number 1, no other rule would aver be checked and all traffic would pass. The opposite (adding a `` deny all `` rule) is also true, then no traffic would get through.
\item The description of a rule. Which once again is quite important in bigger networks.
\item The action of the rule. This can be: drop,reject, accept or inspect. The inspect option is a function that is not used anymore. The other ones are the same as mentioned above.
\item The destination of the packets. In this example the destination of rule 200 is the port group ``TCP-ACCESS``, so all packets that have TCP 80/443 port as destination will match this rule. In rule 100 the address-group ``SCHOOL-NET`` is used. All the packets with one of the IP's that are in that group as destination IP will match this rule. Instead of using groups, it is also possible to assign addresses or ports directly. If for instance only one address may match it is easier to not define a group and just assign the one address here.
\item The source of the packets. The same explanation as for the destination of the packets works here, except of course that the destination is replaced by the source of the packets. In this example the packets need to originate from the local network which is defined by the ``INSIDE-NET`` group.
\item The state established and related lines are for stateful packet filtering. This checks whether a certain package is connected to previously established TCP connections, if this is true, they are allowed through. Or if a new connection is started but the connection is related to an already existing connection.
\end{itemize}
\end{itemize}
Once all these configurations have been made, the only thing left to do is set this rule-set to a certain interface:
\begin{cisco}[title=Assigning the rule-set]
set interfaces ethernet eth0 firewall out name ACCESS-CONTROL
\end{cisco}
Here the interface is specified on which the packets should be checked. As we want to control users from inside our network who are requesting data from outside the network, the most logical choice is to put the rule-set on our outside interface checking the data going OUT. And that is exactly what this line does.
\subsection{Translation to Cisco devices}
To apply this on cisco devices, all one needs to do is define ACL's with the same information as contained in all the groups here. 
\begin{cisco}
access-list 111 permit tcp host 178.62.144.90 172.16.0.0 0.0.255.255 eq www
access-list 111 permit tcp host 178.62.144.90 172.16.0.0 0.0.255.255 eq 443
access-list 111 permit tcp host 178.62.144.90 172.16.0.0 0.0.255.255 eq ftp
interface eth0
ip access-group 110 out
\end{cisco}
This little piece of code allows HTTP requests to pass through the eth0 interface when going out and when they are destined for the website at \textit{178.62.144.90}.  It also allows packets with destination set for port 443(HTTPS) and the ftp protocol. All the rules that we applied in our Vyos router are easily translated to ACL's on a cisco device. It just requires a lot of typing.
\subsection{Effectiveness/conclusion}
So what is the result of these settings? When now connecting to the network the user is able to connect to the domain that we specified (hogent.be) but gets an error whenever trying to connect to any other website. So the goal that was set has been reached. Directly surfing to the IP of the website does not resolve anything as the router does not care how the user connects to a website, it blocks it either way. There are of course ways to circumvent this system (like using a VPN) but the easiest ways of cheating have been prevented. Although nothing has been done against files locally on the computer. It is worth mentioning as well that everything within the own network is still available. Possible file servers can still be reached and function as needed.
\section{Dns Whitelisting}
\subsection{Installing DnsMasq}










%SOURCE VYOS: https://app.vagrantup.com/bertvv/boxes/vyos116