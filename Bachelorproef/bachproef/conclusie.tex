%%=============================================================================
%% Conclusie
%%=============================================================================

\chapter{Conclusion}
\label{ch:conclusie}

%% TODO: Trek een duidelijke conclusie, in de vorm van een antwoord op de
%% onderzoeksvra(a)g(en). Wat was jouw bijdrage aan het onderzoeksdomein en
%% hoe biedt dit meerwaarde aan het vakgebied/doelgroep? Reflecteer kritisch
%% over het resultaat. Had je deze uitkomst verwacht? Zijn er zaken die nog
%% niet duidelijk zijn? Heeft het onderzoek geleid tot nieuwe vragen die
%% uitnodigen tot verder onderzoek?

\section{Comparing the different methods}
The obivous winner of these methods is SEB. It is extremely easy to implement, blocks traffic on way more levels than the other methods do and is very user friendly. And with the correct handling it could be quite unbreakable for students. It is amazing that a tool like this is not being used more frequently. The biggest reason for this is probably because a lot of people simply do not know that it exists. But as big contributors to this software are mostly located in countries who are leading in education worldwide it might not be a bad idea to pick this up.
Upon discovery of this tool the other methods kind off faded in importance and usefulness. As all the other methods only really tackle one issue in the wide range of cheating possibilities. But from the other tested methods DNS whitelisting seems to be the worst one. It has the most loopholes and is harder to implement than the other two are. As well as it has a bigger impact on the network. If a local DNS is already present there will have to be made quite some changes to the configuration. As well as the difficulty  of adding all the IP's and domain names to list. One might argue that a blacklist might be better because there are often offered online. But these will always be one step behind. It's easier to find a certain fish in a lake when only 6 fish are swimming in it, than it is in an ocean where millions of fish are swimming even though a few million have already been pulled out.\\
The ACL's and the firewall are on the same level. If one of these two would be chosen, it's really a matter of preference. The firewall allow for great configuration that goes way farther than simply filtering traffic. But it is once again an extra device that is needed. While a router is present in the network and can do the same functions that are required without having to spend anything new on equipment.  Plus in this thesis the configuration of the router went a lot smoother than the firewall. This can of course depend from system to system, which is way the choice between these really depends on the situation one is in. Both get the job done though. They are a lot easier to implement than DNS whitelisting is. They are more tightly sealed and are more difficult to trick.\\
So all around SEB is the overall winner with ACL's and firewall on a shared second place. A combination of the both could be used as well of course. When certain exams require specific multiple websites to be accessible SEB could be used to prevent the student from using any local documentation while an ACL could block the websites that are not allowed. Doing this one creates a flexible,dynamic and multi functional network in which exams can be made safely.
\section{Issues with the tested methods}
The biggest issue was the fact that a lot of these methods count on the IP or the full qualified domain name of a website. While this is not always as easy to get. A lot of (probably almost all) websites these days using multiple third party components to host their site. This means that simply allowing or blocking one IP-Address / FQDN does not do the trick. When we want to allow a website it might nog load completely or just refuse to load because of certain rules. And when blocking one it might still be accessible to some extend. Figuring out all the information needed to allow one website can be a lot of work and cause a lot of headaches. Furthermore, when the required information is found it might lead to a server on which multiple websites are stored. Trying to block that one website can be very irritating as these websites might change IP's which would lead to us blocking and allowing the wrong sites. Which of course is not in our interest.\\
An other exploit that might come to mind is the use of VPN's. These however could be blocked by blocking certain ports using a firewall/ACL's (which gives these an other advantage above DNS whitelisting). These ports might be quite big in numbers. But they could all be added to a single port group and then added to the block list whenever an exam is conducted. This way no VPN's would be possible but some other functions might malfunction as well at the same time (for some VPN systems for instance once would have to block the SSH port).\\
The last small issue is the fact that Chamillo was not a LSM that is integrated with SEB which limited the testing possibilities. Even though the system is pretty great with a not integrated LSM, it's always good to improve. Maybe someday the Chamillo community will take a look at this and integrate SEB.
\section{Concerns outside of the general environment}
A brief mention should be made to remind people that all the above methods can be made obsolete by some stupid mistakes. If a student manages to use his smartphone all walls are down again. The same goes with god old fashioned cheating using pen and paper. These systems should of course still be prevented by for instance collecting the phones at the start of an examination and handing them back after it. And using only see through pen cases or no pen cases at all. Equally harmful would be an uninformed teacher/overseer. If a student would be able to trick a teacher into giving them information that he should not know, just because the teacher is not someone with great IT knowledge, the whole system once again topples over. To prevent this all teachers and overseers should always be told the information they need. And to not share this information with anyone. Or call someone with the required knowledge when a problem occurs.
\section{Answering the research questions}
\subsubsection{Can we limit access to the internet to only the necessary item?}
A question that is not easily answered. But after this thesis the answer would be yes. Even though students will probably always find a way to circumvent the systems. At the time of deployment there is a possibility to make environments in which only access is given to the required resources. This can be done by using SEB, or ACL's or a combination of the both. While still taking in account the rules outside of the digital world of course (as mentioned in \textit{5.3}.
\subsubsection{Is it possible to set up an exam server, which is only accessible to registered students, where the exams are submitted digitally?}
Even though this question was not anywhere in the spotlight in this thesis, it was mentioned sometimes. The methods still allow all internal servers to be accessed. This was tested by a file server in the test environment. This file server had a share that could be accessed by teachers only, by teachers and students but students could only upload files and a folder in which students could only download. This all stays functional of course after the implementation of the methods as these methods only act on traffic that is willing to leave the traffic. As long as it stays locally nothing will happen to it.
\subsubsection{Can personal applications be filtered while allowing certain necessary programs?}
Yes, as mentioned in the section about SEB, this has functions available exactly for that. There might be other ways possible but research into these seems pointless as SEB does a great job in this.
\subsubsection{Can the setup of this environment be automated so every teacher can easily set up his own exam room?}
Yes. In the part about automation the more technical side was approached \textit{(2.2)}. But once all the necessary files have been created (scripts,roles,conf.seb files) the process becomes pretty easy .With a short and simple explanation most people would probably be able to execute the automation themselves.
\subsubsection{Can this all be done with currently available technology or should the school invest in new equipment/software?}
It depends. If the school wants to use old equipment it would of course work. Nothing that was described in this thesis needs new devices. But maybe in some situations it might be useful to have seperate devices with each it's function. For instance one exam router and one all-round router. This would only require the switch of a cable to implement a whole other system. SEB does not require any new technology whatsoever though. And is supported by all generally used operating systems. Which results in a very low cost implementation.
\section{Possibility for further research}
The integration of SEB into Chamillo would be interesting to further investigate. It make the lives of the teachers a lot easier in our school. Maybe it can be made so that SEB is actually installed on a server in the school and that the students boot it via that server and this way it might add an extra layer of protection. Research around the other methods seems kind of a waste of time as it is clear that there are way more modern and better choices on the market that do much more. The only things that might be interesting is to find a way to effectively find correct IP addresses for certain domains which you want to block that keeps being up to date. So basically, automating the setup of a dynamic whitelist. 
