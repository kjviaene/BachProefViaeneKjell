%%=============================================================================
%% Inleiding
%%=============================================================================

\chapter*{Introduction}
\label{ch:inleiding}

This introduction will give a general idea of what to expect in this thesis. Why it was written, how it was written and what the purpose of it is. 

\section{Issue at hand}
\label{sec:probleemstelling}

At present, a lot of schools (including the HoGent) struggle with setting up environments in which digital examinations can be conducted. This often occurs because of the lack of knowledge around IT or around the general possibilities there are. There are multiple ways to do this and it might not always be clear what way is the one to take. Certain advantages and disadvantage  of those methods are not known as well. That's why in this thesis the most commonly known ways of setting up such an environment are tested and compared with each other. When the thesis has been read the reader will be able to determine what solution fits his needs the best. While also understanding the multiple ways that are viable options. 

\section{Research questions}
\label{sec:onderzoeksvraag}
The specific research questions are as follows:
\begin{itemize}
   \item Can we limit access to the Internet to only the necessary items? (e.g. allowing access to chamilo.hogent.be / netacad.com but not YouTube / Google )
   \item Is it possible to set up an exam server, which is only accessible to registered students, where all the exams are submitted digitally?
   \item Can personal applications be filtered while allowing certain necessary programs?
   \item Can the setup of this environment be automated so every teacher can easily set up his own exam room?
   \item Can this all be done with currently available technology or should the school invest in new equipment/software?
\end{itemize}
These questions are made from the perspective of a student at the HoGent. So it is possible that these questions already have been answered in other schools, but we want to find a solution for schools that do not yet have these answers. Once these have been answered it will be clear what the possibilities are when setting up a protected environment and how easy/difficult this is. What the required resources are and if it is worth it. The actual methods that are executed in this thesis will not answer these questions directly. But it's by testing these methods and looking at the results that we'll be able to get a conclusion out of it.
\section{Research objective}
\label{sec:onderzoeksdoelstelling}
The objective is to be able to start with an unprotected environment and to be able to make a protected one out of it. While taking into account the scenarios in which the environment is located. If a way is found to protect the environment in multiple ways we can see the objective as reached. If however the conclusion is that no method really fulfills our needs, then this thesis has failed its purpose. But even then there will be more clarity around the possible ways to filtering data in a network.
\section{The intention of this thesis}
\label{sec:opzet-bachelorproef}
The thesis consists out of different parts which each have their own purpose. Each part builds towards a common conclusion. To make the structure of this thesis a bit more clear.A short explanation of each chapter follows:
\begin{itemize}
\item In chapter ~\ref{ch:stand-van-zaken} an in depth explanation is given of all the methods that are being tested. After reading this chapter the reader will have a good understanding of how every method works and what the differences are between them on a core level. Based on this information an opinion can already be made as to what method he would prefer.
\item In chapter ~\ref{ch:methodologie} a brief explanation is given as to how these methods will be executed and tested. A short text also discusses the use of automation.
\item In chapter ~\ref{ch:TestEnvironment1} the test environment in which the different methods are tested is explained in more detail. Every device that is used and the basic configuration that it receives are mentioned here.
\item In chapter ~\ref{ch:execution} the actual execution of the methods is done. Here an explanation is given as to how and why certain settings are set. Each method has the same goal (filtering websites) but of course each method has its own way of reaching it. 
\item In chapter ~\ref{ch:conclusie} a conclusion is build on the experiences that were had during the execution. In this part the final decision is made as to what method is the best. This conclusion can vary if a reader finds himself in a different situation than the situation on which this thesis was based.
\end{itemize}


