%%=============================================================================
%% Samenvatting
%%=============================================================================

% TODO: De "abstract" of samenvatting is een kernachtige (~ 1 blz. voor een
% thesis) synthese van het document.
%
% Deze aspecten moeten zeker aan bod komen:
% - Context: waarom is dit werk belangrijk?
% - Nood: waarom moest dit onderzocht worden?
% - Taak: wat heb je precies gedaan?
% - Object: wat staat in dit document geschreven?
% - Resultaat: wat was het resultaat?
% - Conclusie: wat is/zijn de belangrijkste conclusie(s)?
% - Perspectief: blijven er nog vragen open die in de toekomst nog kunnen
%    onderzocht worden? Wat is een mogelijk vervolg voor jouw onderzoek?
%
% LET OP! Een samenvatting is GEEN voorwoord!

%%---------- Nederlandse samenvatting -----------------------------------------
%
% TODO: Als je je bachelorproef in het Engels schrijft, moet je eerst een
% Nederlandse samenvatting invoegen. Haal daarvoor onderstaande code uit
% commentaar.
% Wie zijn bachelorproef in het Nederlands schrijft, kan dit negeren, de inhoud
% wordt niet in het document ingevoegd.

\IfLanguageName{english}{%
\selectlanguage{english}

\chapter*{Summary}
\selectlanguage{english}
}{ }

%%---------- Samenvatting -----------------------------------------------------
% De samenvatting in de hoofdtaal van het document

\chapter*{\IfLanguageName{english}{Summary}{Abstract}}
Setting up a secure environment in any situation is a challenge. With technology constantly evolving, especially in the IT sector, it's always hard to keep security up to date as well. The core of  the purpose in this thesis is to build a system that ensures students who are taking part in an examination only have access to those resources which they are allowed to use. \\ Having a system like this is crucial to having a good grading of courses. When students are able to cheat at examinations as easily as they often can now, all the value of an examination is lost.\\ For the school itself it would seem like the course is too easy and needs to be made harder. Which devalues the material that is being taught in this course.  \\
The students will think that they are having it easy and will pass the exam without any real knowledge of the subject. This may come back in later situations (follow up courses, job inteviews, ...) where they then realize that they seem to be missing some knowledge that they were supposed to have.\\
