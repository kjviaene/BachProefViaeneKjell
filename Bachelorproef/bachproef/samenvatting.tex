%%=============================================================================
%% Samenvatting
%%=============================================================================

% TODO: De "abstract" of samenvatting is een kernachtige (~ 1 blz. voor een
% thesis) synthese van het document.
%
% Deze aspecten moeten zeker aan bod komen:
% - Context: waarom is dit werk belangrijk?
% - Nood: waarom moest dit onderzocht worden?
% - Taak: wat heb je precies gedaan?
% - Object: wat staat in dit document geschreven?
% - Resultaat: wat was het resultaat?
% - Conclusie: wat is/zijn de belangrijkste conclusie(s)?
% - Perspectief: blijven er nog vragen open die in de toekomst nog kunnen
%    onderzocht worden? Wat is een mogelijk vervolg voor jouw onderzoek?
%
% LET OP! Een samenvatting is GEEN voorwoord!

%%---------- Nederlandse samenvatting -----------------------------------------
%
% TODO: Als je je bachelorproef in het Engels schrijft, moet je eerst een
% Nederlandse samenvatting invoegen. Haal daarvoor onderstaande code uit
% commentaar.
% Wie zijn bachelorproef in het Nederlands schrijft, kan dit negeren, de inhoud
% wordt niet in het document ingevoegd.

\IfLanguageName{english}{%
\selectlanguage{english}

\chapter*{Samenvatting (dutch)}
\selectlanguage{english}
}{ }
Het opzetten van een beveiligde digitale omgeving is altijd een uitdaging, ongeacht de situatie. Door de continue ontwikkeling van technologie, en in het bijzonder in het vakgebied van de informatica, is het geen evidentie om de beveiliging even snel te doen mee evolueren. Het onderwerp van deze bachelorproef is het voorzien van een specifieke beveiliging, het opzetten van een omgeving waarin studenten een digitaal examen kunnen afleggen. \\

We kiezen ervoor de toegang tot bepaalde middelen te beperken om zo de kans op spieken te verminderen. Het beveiligen tegen spieken is in een dergelijke omgeving cruciaal want anders zijn de resultaten van testen en examens onbetrouwbaar. Een examen waar makkelijk kan worden gespiekt, verliest zijn nut. In dergelijk geval geven de resultaten niet weer wat we willen meten, namelijk de aanwezige kennis, zonder het gebruiken van externe bronnen. \\
Geen betrouwbare resultaten leiden tot mogelijk verkeerde conclusies en ondergraven de algemene kwaliteit van een opleiding. Het ligt voor de hand dat elke opleider dit wil vermijden. \\

In het kader van deze proef werd onderzoek gedaan naar de diverse mogelijke manieren waarop men een beveiligde netwerkomgeving kan opzetten voor het afleggen van digitale examens. 
De onderzochte methodes zijn: 
\begin{itemize}
\item Het filteren op basis van een DNS Whitelist
\item Het filteren op basis van een ACL op een router
\item Het filteren op basis van een firewall
\item Het filteren a.d.h.v. een applicatie
\end{itemize}
Dit zijn de meest voor de hand liggende methoden die gebruikt worden om informatie te filteren in een netwerk. Er wordt onderzocht welke van deze methoden goed zijn voor deze toepassing en hoe goed ze zijn in wat ze doen. Ook de eenvoud van het configureren en de kostprijs worden in rekening gebracht. \\
Uit het onderzoek leren we dat DNS whitelisting erg veel werk vereist en te gemakkelijk omzeild kan worden, dus de facto niet bruikbaar is. ACL en een firewall hebben dit minder en zijn daarom betere keuzes. Het gebruik maken van een applicatie die lokaal op elke computer staat geïnstalleerd, is echter duidelijk de beste keuze die kan gemaakt worden. We spreken hier van Safe Exam Browser (SEB). We kwamen deze toepassing op het spoor via de literatuurstudie van een thesis.
\\
Deze tool bleek uitstekend te werken en beantwoordde aan de gestelde verwachtingen. We bereikten vlot alle gewenste effecten en ook was het gebruik heel gemakkelijk. Er is echter wel voorlopig nog één nadeel: de tool is niet geïntegreerd met chamilo. Dit kan echter onderwerp zijn van verder onderzoek.



%%---------- Samenvatting -----------------------------------------------------
% De samenvatting in de hoofdtaal van het document

\chapter*{\IfLanguageName{english}{Summary}{Abstract}}
Setting up a secure environment in any situation is a challenge. With technology constantly evolving, especially in the IT sector, it's always hard to keep security up to date. The core purpose of this thesis is to build a system that ensures students who are taking part in a digital examination only have access to those resources which they are allowed to use. \\ Having a system like this is crucial to having a good grading system. And to minimize the chances that a student cheated. When students are able to cheat at examinations as easily as they often can now, all the value of an examination is lost.\\ For the school itself it would seem like the course is too easy and needs to be made harder. Which devalues the material that is being taught in this course.  \\
The students will think that the course is no challenge and will pass the exam without any real knowledge of the subject. This may come back to bite them in situations later on(follow up courses, job interviews, ...), where they then realize that they seem to be missing some knowledge that they were supposed to have.\\
Multiple methods were tested to find a system in which these problems are resolved. These methods are:
\begin{itemize}
\item Filtering using a DNS Whitelist
\item Filtering using an ACL on a router
\item Filtering using a firewall
\item Filtering using an application.
\end{itemize}
These are the most common ways of filtering internet access in a network. Research is done as to what method is good and how good these methods are compared with each other. How complicated they are to set up and maintain. After testing these methods a conclusion was made that DNS whitelisting is the worst choice. This method requires the most work and has the lowest performance/quality rating. The router ACL method and setting up a firewall are pretty matching in performance and quality. They are easy to configure and allow for better protection. The firewall has more possibilities but can be harder to set up while the router is a bit simpler but is easy to set up. The best choice however is without a doubt the application (Safe Exam Browser). This application secures every device on its own and allows for local filtering of the internet and the local files. This is the perfect solution to our issue. It's easy to use and performs really well. The only downside to it is that there is no integration with Chamilo. It can still be used but some authorization methods are not applicable. This however could be the subject of an other research paper.